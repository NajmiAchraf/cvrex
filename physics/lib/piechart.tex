%-----------------------------------------------------------------------------------------------------------------------------------------------%
%	MIT License
%	
%	Copyright (c) 2021 ACHRAF NAJMI
%	
%	Permission is hereby granted, free of charge, to any person obtaining a copy
%	of this software and associated documentation files (the "Software"), to deal
%	in the Software without restriction, including without limitation the rights
%	to use, copy, modify, merge, publish, distribute, sublicense, and/or sell
%	copies of the Software, and to permit persons to whom the Software is
%	furnished to do so, subject to the following conditions:
%	
%	The above copyright notice and this permission notice shall be included in all
%	copies or substantial portions of the Software.
%	
%	THE SOFTWARE IS PROVIDED "AS IS", WITHOUT WARRANTY OF ANY KIND, EXPRESS OR
%	IMPLIED, INCLUDING BUT NOT LIMITED TO THE WARRANTIES OF MERCHANTABILITY,
%	FITNESS FOR A PARTICULAR PURPOSE AND NONINFRINGEMENT. IN NO EVENT SHALL THE
%	AUTHORS OR COPYRIGHT HOLDERS BE LIABLE FOR ANY CLAIM, DAMAGES OR OTHER
%	LIABILITY, WHETHER IN AN ACTION OF CONTRACT, TORT OR OTHERWISE, ARISING FROM,
%	OUT OF OR IN CONNECTION WITH THE SOFTWARE OR THE USE OR OTHER DEALINGS IN THE
%	SOFTWARE.
%
%-----------------------------------------------------------------------------------------------------------------------------------------------


%counters for chart loop
\newcounter{a}
\newcounter{b}
\newcounter{c}

% draw a slice for a chart
% param 1: Circle form - 90 = quarter, 180 = half, 360 = full
% param 2: scale default=1 (scales only chart, not label text)
% param 3: border color
% param 4: label text color
% param 5: label bg color
% param 6:
\newenvironment{piechart}[5] {

	% draw a slice for a chart
	% param 1: value x of 100
	% param 2: label text
	% param 3: fill color
	% param 4:
	% param 5:
	% param 6:
	\newcommand{\slice}[3] {

		\setcounter{a}{\value{b}}
		\addtocounter{b}{##1}

		%set from angle point
		\pgfmathparse{\thea/100*#1}
	  	\let\pointa\pgfmathresult

		%set toanglepoint
		\pgfmathparse{\theb/100*#1}
	  	\let\pointb\pgfmathresult

		%set midangle
	 	\pgfmathparse{0.5*\pointa+0.5*\pointb}
	  	\let\midangle\pgfmathresult
		
		% draw the slice
	  	\filldraw[fill=##3!100,draw=#3!100, line width=2pt ] (0,0) -- (\pointa:#2) arc (\pointa:\pointb:#2) -- cycle;

	  	% draw label
	  	\node[label=\midangle:\colorbox{#5}{\textcolor{#4}{##2}}] at (\midangle:#2) {};

		\filldraw[fill=#3,draw=none] (0,0) circle (#2/2);
	}

	% execute commands
	\setcounter{a}{0}
	\setcounter{b}{0}
	\begin{tikzpicture}
}
{\end{tikzpicture}}

% --------------------------------------------------------------------------------

\newcommand{\progressarc}[8]{
\begin{tikzpicture}
\draw [line width=#1,#4] (0,#3)  arc[start angle=90,radius=#3,delta angle=-3.60*100];  % von Prozent zu Kreiswerten
\draw [line width=#1,#2] (0,#3)  arc[start angle=90,radius=#3,delta angle=-3.60*#8];  % von Prozent zu Kreiswerten
\node[] at (0,0) {#5};
\node[#7=#3+#1,align=center] at (0,0) {#6};
\end{tikzpicture}
}