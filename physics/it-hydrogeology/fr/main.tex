%-----------------------------------------------------------------------------------------------------------------------------------------------%
%	MIT License
%	
%	Copyright (c) 2021 ACHRAF NAJMI
%	
%	Permission is hereby granted, free of charge, to any person obtaining a copy
%	of this software and associated documentation files (the "Software"), to deal
%	in the Software without restriction, including without limitation the rights
%	to use, copy, modify, merge, publish, distribute, sublicense, and/or sell
%	copies of the Software, and to permit persons to whom the Software is
%	furnished to do so, subject to the following conditions:
%	
%	The above copyright notice and this permission notice shall be included in all
%	copies or substantial portions of the Software.
%	
%	THE SOFTWARE IS PROVIDED "AS IS", WITHOUT WARRANTY OF ANY KIND, EXPRESS OR
%	IMPLIED, INCLUDING BUT NOT LIMITED TO THE WARRANTIES OF MERCHANTABILITY,
%	FITNESS FOR A PARTICULAR PURPOSE AND NONINFRINGEMENT. IN NO EVENT SHALL THE
%	AUTHORS OR COPYRIGHT HOLDERS BE LIABLE FOR ANY CLAIM, DAMAGES OR OTHER
%	LIABILITY, WHETHER IN AN ACTION OF CONTRACT, TORT OR OTHERWISE, ARISING FROM,
%	OUT OF OR IN CONNECTION WITH THE SOFTWARE OR THE USE OR OTHER DEALINGS IN THE
%	SOFTWARE.
%
%-----------------------------------------------------------------------------------------------------------------------------------------------


\documentclass[darkxp]{../../lib/physics}
\usepackage{../../lib/physics}
% available options are: darkpython, lightpython, darkxp, lightxp, vista, black&white, dark, light
\usepackage[utf8]{inputenc}
\usepackage[default]{raleway}
\usepackage[margin=1cm, a4paper]{geometry}


%----------------------------------------------------------------------------------------
% 	PIE CHART
%----------------------------------------------------------------------------------------
%-----------------------------------------------------------------------------------------------------------------------------------------------
%
%	MIT License
%	
%	Copyright (c) 2021 ACHRAF NAJMI
%	
%	Permission is hereby granted, free of charge, to any person obtaining a copy
%	of this software and associated documentation files (the "Software"), to deal
%	in the Software without restriction, including without limitation the rights
%	to use, copy, modify, merge, publish, distribute, sublicense, and/or sell
%	copies of the Software, and to permit persons to whom the Software is
%	furnished to do so, subject to the following conditions:
%	
%	The above copyright notice and this permission notice shall be included in all
%	copies or substantial portions of the Software.
%	
%	THE SOFTWARE IS PROVIDED "AS IS", WITHOUT WARRANTY OF ANY KIND, EXPRESS OR
%	IMPLIED, INCLUDING BUT NOT LIMITED TO THE WARRANTIES OF MERCHANTABILITY,
%	FITNESS FOR A PARTICULAR PURPOSE AND NONINFRINGEMENT. IN NO EVENT SHALL THE
%	AUTHORS OR COPYRIGHT HOLDERS BE LIABLE FOR ANY CLAIM, DAMAGES OR OTHER
%	LIABILITY, WHETHER IN AN ACTION OF CONTRACT, TORT OR OTHERWISE, ARISING FROM,
%	OUT OF OR IN CONNECTION WITH THE SOFTWARE OR THE USE OR OTHER DEALINGS IN THE
%	SOFTWARE.
%
%-----------------------------------------------------------------------------------------------------------------------------------------------


%counters for chart loop
\newcounter{a}
\newcounter{b}
\newcounter{c}

% draw a slice for a chart
% param 1: Circle form - 90 = quarter, 180 = half, 360 = full
% param 2: scale default=1 (scales only chart, not label text)
% param 3: border color
% param 4: label text color
% param 5: label bg color
% param 6:
\newenvironment{piechart}[5] {

	% draw a slice for a chart
	% param 1: value x of 100
	% param 2: label text
	% param 3: fill color
	% param 4:
	% param 5:
	% param 6:
	\newcommand{\slice}[3] {

		\setcounter{a}{\value{b}}
		\addtocounter{b}{##1}

		%set from angle point
		\pgfmathparse{\thea/100*#1}
	  	\let\pointa\pgfmathresult

		%set toanglepoint
		\pgfmathparse{\theb/100*#1}
	  	\let\pointb\pgfmathresult

		%set midangle
	 	\pgfmathparse{0.5*\pointa+0.5*\pointb}
	  	\let\midangle\pgfmathresult
		
		% draw the slice
	  	\filldraw[fill=##3!100,draw=#3!100, line width=2pt ] (0,0) -- (\pointa:#2) arc (\pointa:\pointb:#2) -- cycle;

	  	% draw label
	  	\node[label=\midangle:\colorbox{#5}{\textcolor{#4}{##2}}] at (\midangle:#2) {};

		\filldraw[fill=#3,draw=none] (0,0) circle (#2/2);
	}

	% execute commands
	\setcounter{a}{0}
	\setcounter{b}{0}
	\begin{tikzpicture}
}
{\end{tikzpicture}}


%----------------------------------------------------------------------------------------
% 	TIMELINE VERTICAL & HORIZONTAL CHART
%----------------------------------------------------------------------------------------
%-----------------------------------------------------------------------------------------------------------------------------------------------
%
%	MIT License
%	
%	Copyright (c) 2021 ACHRAF NAJMI
%	
%	Permission is hereby granted, free of charge, to any person obtaining a copy
%	of this software and associated documentation files (the "Software"), to deal
%	in the Software without restriction, including without limitation the rights
%	to use, copy, modify, merge, publish, distribute, sublicense, and/or sell
%	copies of the Software, and to permit persons to whom the Software is
%	furnished to do so, subject to the following conditions:
%	
%	The above copyright notice and this permission notice shall be included in all
%	copies or substantial portions of the Software.
%	
%	THE SOFTWARE IS PROVIDED "AS IS", WITHOUT WARRANTY OF ANY KIND, EXPRESS OR
%	IMPLIED, INCLUDING BUT NOT LIMITED TO THE WARRANTIES OF MERCHANTABILITY,
%	FITNESS FOR A PARTICULAR PURPOSE AND NONINFRINGEMENT. IN NO EVENT SHALL THE
%	AUTHORS OR COPYRIGHT HOLDERS BE LIABLE FOR ANY CLAIM, DAMAGES OR OTHER
%	LIABILITY, WHETHER IN AN ACTION OF CONTRACT, TORT OR OTHERWISE, ARISING FROM,
%	OUT OF OR IN CONNECTION WITH THE SOFTWARE OR THE USE OR OTHER DEALINGS IN THE
%	SOFTWARE.
%
%-----------------------------------------------------------------------------------------------------------------------------------------------


%----------------------------------------------------------------------------------------
%   TIMELINE VERTICAL CHART
%----------------------------------------------------------------------------------------
\newcounter{yearcount}
\newcounter{leftcount}
% env cvtimeline
%
% creates a vertical cv timeline
%
% param 1: start year
% param 2: end year
% param 3: overall width
% param 4: overall height             {\parbox{\myboxwidth}
% param 5: level (width)
\newenvironment{timelinevertical}[6]{
    \newcommand{\cvcategory}[2]{
        \node[label=\mbox{\colorbox{##1}{\strut\hspace{2pt}}\colorbox{white}{\textcolor{textcol}{##2}}}] at (0,-5) {}; % start year
    }
    \newcommand{\bxwidth}{4.5}
    \newcommand{\bxheight}{2}
    % creates a stretched box as cv entry headline followed by two paragraphs about 
    % the work you did
    % param 1:  event start month/year
    % param 2:  event end month/year
    % param 3:  event name
    % param 4:  event position x
    % param 5:  event position y
    % param 6:  color
    % param 7:  level (position, use minus for the opposite placement) {give the param 7 (-#5) to get null value}
    \newcommand{\cvevent}[8] {
        \foreach \monthf/\yearf in {##1} {
            \foreach \montht/\yeart in {##2} {
                \pgfmathparse{#3/\fullrange*((\yearf-#1)+(\monthf/12))}
                \let\startexp\pgfmathresult
                \pgfmathparse{#3/\fullrange*((\yeart-#1)+(\montht/12))}
                \let\endexp\pgfmathresult
                \pgfmathparse{1/(\endexp-\startexp+1)}
                \let\lenexp\pgfmathresult
                \pgfmathparse{0.5*\endexp+0.5*\startexp}
                \let\midexp\pgfmathresult
                
                \filldraw[fill=##6, draw=none] (-##7, \startexp) rectangle (-#5-##7, \endexp);% rectangle, opacity=0.9
                	
				\draw[draw=##6, line width=1.5pt, anchor=west] (0, \startexp) -- (-#5-##7, \startexp);% ex-line
                
				\draw[draw=##6, line width=1.5pt, anchor=west] (0, \startexp) -- (##4, \startexp+##5);% line
				
				\node[fill=##6, right=-0.1em, inner sep=0.5em, label={[label distance=-4]0:\colorbox{sectcol}{\parbox{0.75\textwidth}{\textcolor{gray}{##1}\hspace{3pt}\textcolor{textcol}{##3}}}}] at (##4, \startexp+##5) {};% textbox	\textcolor{##6}{|}		
				
				
				% \node[below, inner sep=1.5em] at (\eventposition,0) {##1};%draw year below

				\draw[fill=white, draw=##6, line width=0.1em]  (0,\startexp) circle (0.1);%year circle
				
            }
            \addtocounter{leftcount}{1}
        }
    }

    %--------------------------------------------------------------------------------------
    %   BEGIN
    %--------------------------------------------------------------------------------------

    \begin{tikzpicture}
    \setcounter{leftcount}{1}
    %calc fullrange= number of years
    \pgfmathparse{(#2-#1)}
    \let\fullrange\pgfmathresult
    \draw[draw=textcol,line width=4pt] (0,0) -- (0,#3) ;    %the timeline
    %for each year put a horizontal line in place
    \setcounter{yearcount}{1}
    \whiledo{\value{yearcount} < \fullrange}{
        \draw[fill=white,draw=textcol, line width=2pt]  (0,#3/\fullrange*\value{yearcount}) circle (0.2);
        \stepcounter{yearcount}
    }
    %start year
    \filldraw[fill=white!100,draw=textcol,line width=3pt] (0,-0.5) circle (0.5);
    \node[label=\textcolor{textcol}{\textbf{\small#1}}] at (0,-0.85) {}; 
    %end year
    \filldraw[fill=white!100,draw=textcol,line width=5pt] (0,#3+0.75) circle (0.75);
    \node[label=\textcolor{textcol}{\textbf{\large#2}}] at (0,#3+0.42) {}; 
}%end begin part of newenv
{\end{tikzpicture}}


%----------------------------------------------------------------------------------------
%   TIMELINE HORIZONTAL CHART
%----------------------------------------------------------------------------------------
% \newcounter{yearcount}
% \newcounter{leftcount}
% env cvtimeline
%
% creates a vertical cv timeline
%
% param 1: start year
% param 2: end year
% param 3: overall width
% param 4: overall height             {\parbox{\myboxwidth}
% param 5: level (width)
\newenvironment{timelinehorizontal}[5]{
    \newcommand{\cvcategory}[2]{
        \node[label=\mbox{\colorbox{##1}{\strut\hspace{2pt}}\colorbox{white}{\textcolor{textcol}{##2}}}] at (0,-5) {}; % start year
    }
    \newcommand{\bxwidth}{4.5}
    \newcommand{\bxheight}{2}
    % creates a stretched box as cv entry headline followed by two paragraphs about 
    % the work you did
    % param 1:  event start month/year
    % param 2:  event end month/year
    % param 3:  event name
    % param 4:  event position y
    % param 5:  event position x
    % param 6:  color
    % param 7:  level (position, use minus for left placement)
    \newcommand{\cvevent}[8] {
        \foreach \monthf/\yearf in {##1} {
            \foreach \montht/\yeart in {##2} {
                \pgfmathparse{-#3/\fullrange*((\yearf-#1)+(\monthf/12))}
                \let\startexp\pgfmathresult
                \pgfmathparse{-#3/\fullrange*((\yeart-#1)+(\montht/12))}
                \let\endexp\pgfmathresult
                \pgfmathparse{1/(\endexp-\startexp+1)}
                \let\lenexp\pgfmathresult
                \pgfmathparse{0.5*\endexp+0.5*\startexp}
                \let\midexp\pgfmathresult
                
                \pgfmathparse{-#3/\fullrange*((\yearf-#1)+(\monthf/12))}% get the event position
                \let\eventposition\pgfmathresult
                
                \filldraw[fill=##6, draw=none, opacity=0.9] (\startexp ,0.15+##7) rectangle (\endexp, 0.15+##7+#5);
                
				\draw[draw=##6, line width=1.5pt, anchor=west] (\eventposition,0) -- (\eventposition+##5,##4);% line			
				
				\node[fill=##6,right=-0.1em,inner sep=0.5em, label={[label distance=0]0:\colorbox{sectcol}{\textcolor{textcol}{##3}}}] at (\eventposition+##5,##4) {\textcolor{white}{##1}};% textbox			
				
				% \node[below, inner sep=1.5em] at (\eventposition,0) {##1};%draw year below

				\draw[fill=white,draw=##6, line width=0.1em]  (\eventposition,0) circle (0.1);%year circle
            }
            \addtocounter{leftcount}{1}
        }
    }

    %--------------------------------------------------------------------------------------
    %   BEGIN
    %--------------------------------------------------------------------------------------

    \begin{tikzpicture}
    \setcounter{leftcount}{1}
    %calc fullrange= number of years
    \pgfmathparse{(#2-#1)}
    \let\fullrange\pgfmathresult
    \draw[draw=textcol,line width=4pt] (0,0) -- (-#3,0) ;    %the timeline
    
    %for each year put a horizontal line in place
    \setcounter{yearcount}{1}
    \whiledo{\value{yearcount} < \fullrange}{
        \draw[fill=white,draw=textcol, line width=2pt]  (-#3/\fullrange*\value{yearcount},0) circle (0.2);
        \stepcounter{yearcount}
    }
	
	%start year
    \filldraw[fill=white!100,draw=textcol,line width=3pt] (0,0) circle (0.5);
    \node[draw=none, label=\textcolor{textcol}{\textbf{\small#1}}] at (0,-0.35) {}; 
    %end year
    \filldraw[fill=white!100,draw=textcol,line width=5pt] (-#3,0) circle (0.75);
    \node[draw=none, label=\textcolor{textcol}{\textbf{\large#2}}] at (-#3,-0.35) {}; 
    
    
}%end begin part of newenv
{\end{tikzpicture}}

%------------------------------------------------------------------ Variablen



\newlength{\rightcolwidth}
\newlength{\leftcolwidth}
\setlength{\leftcolwidth}{0.3\textwidth}
\setlength{\rightcolwidth}{0.65\textwidth}

%------------------------------------------------------------------
\title{cvrex}
\author{\LaTeX{} NajmiAchraf}
\date{September 2021}

\pagestyle{empty}
\begin{document}


\thispagestyle{empty}
%-------------------------------------------------------------

\section*{Start}



\header{\bgupper{headerfontbox}{headerfontboxfont}{\bfseries\Huge Noura Najmi}}{\bg{headerfontbox}{headerfontboxfont}{\large Hydro-Informaticienne}}{\bg{headerfontbox}{headerfontboxfont}{\small 
\begin{minipage}[t]{0.555\textwidth}
Je suis une Hydro-Informaticienne avec une expérience de travail dans des projets de recherche. 
Je suis toujours à la recherche de nouveaux projets interdisciplinaires passionnants liés au domaine de l'eau.
Disponible pour un stage PFE
\end{minipage}}}{../pic/me.JPG}{headerblue}{3.5cm}{1cm}



%------------------------------------------------

% the "invisible" heading has to be in here, otherwise it won't start the Paracols ... strange ...
\subsection*{}
\vspace{4em}

\setlength{\columnsep}{1.5cm}
\columnratio{0.3}[0.65]
\begin{paracol}{2}
\hbadness5000
%\backgroundcolor{c[1]}[rgb]{1,1,0.8} % cream yellow for column-1 %\backgroundcolor{g}[rgb]{0.8,1,1} % \backgroundcolor{l}[rgb]{0,0,0.7} % dark blue for left margin

\paracolbackgroundoptions

% 0.9,0.9,0.9 -- 0.8,0.8,0.8

\vspace{-2em}
\footnotesize
{\setasidefontcolour
\bgupper{cvyellow}{iconcolour}{Faits} \\
\bg{cvyellow}{iconcolour}{Personnel} \\

\begin{tabular}{c @{\hspace{0.5em}} l | l}
\faBirthdayCake & Née le &  22/05/1997 \\
\faFemale & Sexe & Femelle \\
% \faGlobe & Nationality & Moroccan \\
\faMapMarker & Emplacement & Casablanca, Maroc \\
\end{tabular}

\bigskip

\bgupper{cvyellow}{iconcolour}{Competences} \\
\bg{cvyellow}{iconcolour}{Domaines de Compétences}\\

% PIE CHART	
\begin{piechart}{360}{1}{bgchart}{iconcolour}{skilllabelcolour}\hspace{-1.5em}
	\slice{33}{Modélisation}{cvyellow}
	\slice{27}{Simulation}{cvred}
	\slice{23}{Cartographie}{headerblue}
	\slice{17}{Développement}{cvorange}
\end{piechart}

\bigskip

\bg{cvyellow}{iconcolour}{Langues} \\

\begin{minipage}[t]{\leftcolwidth}
\begin{tabular}{c @{\hspace{0.5em}} l | l}
\faLanguage & \textbf{Français} & B2 \pictofraction{\faCircle}{cvpurple}{4}{black!30}{2}{\tiny}\\
\faLanguage & \textbf{Anglais} & A2 \pictofraction{\faCircle}{cvpurple}{2}{black!30}{4}{\tiny}
\end{tabular}
\end{minipage}

\bigskip

\bgupper{cvyellow}{iconcolour}{Specialite} \\
\bg{cvyellow}{iconcolour}{Environnement et Eau} \\

\begin{tabular}{c @{\hspace{0.5em}} l | c @{\hspace{0.5em}} l}% \faAngleDoubleRight
\faIcon{water} & Hydraulique & \faIcon{city} & Assainissement  \\
\faIcon{globe-africa} & Hydrogéologie & & urbain \\
\faIcon{tint} & Hydrologie & \faIcon{hand-holding-water} & Analyse et trai- \\ 
\faIcon{satellite} & Télédétection & & -tement de l’eau \\
\faIcon{hard-hat} & Géotechnique &  \faIcon{map-marked-alt} & SIG
\end{tabular}

\bigskip

\bgupper{cvyellow}{iconcolour}{Informatique} \\
\bg{cvyellow}{iconcolour}{Programmes Logiciels} \\

\bubblediagram{{\icon{\Large\faLaptop}{fontdiagram}{\Large} \\ \textbf{\color{fontdiagram}{ArcGIS}}}, Eclipse \\IDE, GMS \\Aquaveo, Code \\Block, HEC-RAS, Adobe \\Illustrator}

% \bubblediagram{{\icon{\Large\faCode}{fontdiagram}{\Large} \\ \textbf{\color{fontdiagram}{\Large C}}}, \large Matlab, \large Java}

\bigskip


\bg{cvyellow}{iconcolour}{Système d'Exploitation}\\

\icon{\faWindows}{labelcolour}{\Large}\color{labelcolour}{\large Windows}

\bigskip
% \smallskip

\bg{cvyellow}{iconcolour}{Langages de Programmation} \\

\begin{minipage}[t]{0.3\textwidth} % \vfill
\begin{tabular}{r @{\hspace{0.5em}}l}
     \bg{skilllabelcolour}{iconcolour}{\faChartArea \hspace{0.1em} Matlab} & \barrule{0.4}{0.5em}{materialcyan} \\
     \bg{skilllabelcolour}{iconcolour}{\faIcon{java} \hspace{0.1em} Java} & \barrule{0.3}{0.5em}{materialorange} \\
     \bg{skilllabelcolour}{iconcolour}{\faIcon{copyright} \hspace{0.1em} Langage C} & \barrule{0.2}{0.5em}{materialindigo} \\
     % \bg{skilllabelcolour}{iconcolour}{HEC-RAS} & \barrule{0.3}{0.5em}{materialteal} \\
     % \bg{skilllabelcolour}{iconcolour}{Eclipse IDE} & \barrule{0.25}{0.5em}{materiallime} \\
     % \bg{skilllabelcolour}{iconcolour}{GMS Aquaveo} & \barrule{0.2}{0.5em}{materiallime} \\
     % \bg{skilllabelcolour}{iconcolour}{\faCodeBranch \hspace{0.1em} \faGit} & \barrule{0.15}{0.5em}{materialamber} \\
    
\end{tabular}

% \begin{minipage}[t]{0.3\textwidth} % \vfill
% \begin{tabular}{r @{\hspace{0.5em}}l}
     % \bg{skilllabelcolour}{iconcolour}{ArcGIS} & \barrule{0.45}{0.5em}{materiallime} \\
     % \bg{skilllabelcolour}{iconcolour}{Code Block} & \barrule{0.4}{0.5em}{materialteal} \\
     % \bg{skilllabelcolour}{iconcolour}{Adobe Illustrator} & \barrule{0.35}{0.5em}{materialorange} \\
     % \bg{skilllabelcolour}{iconcolour}{HEC-RAS} & \barrule{0.3}{0.5em}{materialcyan} \\
     % \bg{skilllabelcolour}{iconcolour}{Eclipse IDE} & \barrule{0.25}{0.5em}{materialindigo} \\
     % \bg{skilllabelcolour}{iconcolour}{GMS Aquaveo} & \barrule{0.2}{0.5em}{materiallime} \\
     % \bg{skilllabelcolour}{iconcolour}{\faCodeBranch \hspace{0.1em} \faGit} & \barrule{0.15}{0.5em}{materialamber} \\
    
% \end{tabular}

% \dashrule{}{}
\vspace{-3em}
\end{minipage}


\phantom{turn the page}

\phantom{turn the page}
}
%-----------------------------------------------------------

\switchcolumn

\small
\vspace{-2em}
\section*{Education}
\begin{timelinehorizontal}{2018}{2022}{11}{black}
			
			\cvevent{9/2019}{4/2021}{Master Spécialisé : Hydroinformatique et Gestion des Hydrosystèmes}{1.8}{-5.1}{cvyellow}{iconcolour}{-0.7}
	
			\cvevent{9/2018}{6/2019}{Licence : Physique Mécanique Énergétique}{1.1}{0}{headerblue}{iconcolour}{-0.7}
			
\end{timelinehorizontal}

% \vspace{-0.5em}

% \begin{minipage}[t]{0.4\textwidth}
\section*{Diplômes}
\begin{tabular}{r| p{0.45\textwidth} c}

    \cvschool{}{Master Spécialisé (MS) \color{cvred}}{Université Ibn Tofail de Kénitra}{Faculté des Sciences\color{headerblue}}{Spécialité : Hydroinformatique et Gestion des Hydrosystèmes}{../pic/univitkfs.png} \\

    \cvdegree{2019}{Licence d'Études Fondamentales (LEF) \color{cvred}}{Université Hassan II de Casablanca}{FSBM \color{headerblue}}{Filière : Science Matière Physique | Parcours : Mécanique Énergétique}{../pic/univh2fsbm.png} \\

    \cvdegree{2018}{Diplôme d'Études Universitaires Générales (DEUG) \color{cvred}}{Université Hassan II de Casablanca}{FSBM \color{headerblue}}{Filière : Science Matière Physique}{../pic/univh2fsbm.png} \\

    \cvschool{2015}{Diplôme du Baccalauréat \color{cvred}}{Lycée}{Ibnou Zaidone \color{headerblue}}{Série : Sciences Expérimentales | Option : Physique}{../pic/minister.jpg}
\end{tabular}
% \end{minipage}\hfill

% \vspace{-0.5em}
\section*{Projets}
\begin{tabular}{r| p{0.45\textwidth} c}
    
    \cvproject{2019}{Projet de Fin d'Études}{UHIIC}{Faculté des Sciences Ben M'sick \color{cvyellow}}{Modélisation du Graphène par la Mécanique Moléculaire des Structures}{../pic/univh2fsbm.png} \\
    
    \cvproject{2019}{Projet Entreprenariat}{UHIIC}{Faculté des Sciences Ben M'sick \color{cvyellow}}{Projet de Création d'un produit de sécurité humaine : Détecteur des fuites de gaz "iProtection"}{../pic/iPROTECTION.png}
    
\end{tabular}

\begin{minipage}[t]{0.3\textwidth}
\vspace{1em}
\section*{Logiciels}
\textcolor{black}{Actuellement, en temps réel je développe des programmes et je crée des interfaces graphiques basiques avec Java, en tout ce qui concerne le domaine de l'hydrologie et l'hydrogéologie.}

\end{minipage}\hfill
\begin{minipage}[t]{0.3\textwidth}
\vspace{1em}
\section*{Programmation}
\textcolor{black}{Pour mes projets de modélisation, les systèmes d'information géographique SIG et le modèle numériques du terrain MNT, j'utilise régulièrement des logiciels qui se présentent sous : Matlab, ArcGis, HEC-RAS, Plaxis.}

\end{minipage}

\begin{minipage}[t]{0.65\textwidth}
\vspace{1em}
\section*{Activites}
\mbox{
	\parbox[b][2cm][c]{7cm}{
		\progressarc{0.55em}{cvyellow}{0.6cm}{timelinebg}{\huge\color{black}\faPlayCircle}{Tutoriels \\}{below}{82.5} \hfill
		\progressarc{0.55em}{cvyellow}{0.6cm}{timelinebg}{\huge\color{black}\faNewspaper}{Bouquiner \\}{below}{75} \hfill
		\progressarc{0.55em}{cvyellow}{0.6cm}{timelinebg}{\huge\color{black}\faChess}{Jeux \\Cérébrale}{below}{67.5} \hfill
		\progressarc{0.55em}{cvyellow}{0.6cm}{timelinebg}{\huge\color{black}\faBicycle}{Sport \\}{below}{60}
	}

	\parbox[b][2.2cm][c]{5cm}{
			\textcolor{black}{Parmis les dominantes activités que je pratique est de naviguer sur internet à la recherche de nouvelles concernant mon domaine d'étude et entraîné sur tous les tutoriels qui m'intéresse.}
		}
	}

\end{minipage}









\vfill{} % Whitespace before final footer

%----------------------------------------------------------------------------------------
%	FINAL FOOTER
%----------------------------------------------------------------------------------------
\setlength{\parindent}{0pt}
\begin{minipage}[t]{\rightcolwidth}
\begin{center}\fontfamily{\sfdefault}\selectfont \color{black!70}
{\href{callto:+212658525391}{\icon{\faMobile}{black}{} \textcolor{black!70}{+212 6 58 52 53 91}}\href{https://www.linkedin.com/in/nora-najmi-027344177}{\icon{\faLinkedin}{blue}{} \textcolor{black!70}{linkedin.com/in/nora-najmi-027344177}} 
}\\

{\href{mailto:noura.najmi@uit.ac.ma}{\icon{\faAt}{red}{} \textcolor{black!70}{noura.najmi@uit.ac.ma}}\href{https://www.facebook.com/noora.sweet.5851}{\icon{\faFacebookSquare}{blue}{} \textcolor{black!70}{facebook.com/noora.sweet.5851}}

}
\end{center}\vspace{-0.5em} % \vspace{-2em}
\end{minipage}


\end{paracol}
\end{document}
